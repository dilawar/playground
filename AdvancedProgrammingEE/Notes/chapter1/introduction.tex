%% This is our first chapter, section 1.
\chapter{Introduction}
\label{chap1:introduction}
Fortunately a computer does not have a common sense. At best, it can do whatever it is told to. Nothing more, nothing less. We can formulate an algorithm describing how a problem is to be solved. Our algorithm, when implemented on a computer, may not give desirable results. The reasons could be many. First and foremost, our algorithm is faulty. It has a logical error. Computer's decision making abilities depends on the logic provided in an algorith. We can do \textit{testing} to check whether our algorithm is correct as expected or not. Testing in itself a very costly task. It requires a lot of thinking about what combination of input will give what output. Sometimes an exhaustibe testing may be very costly. For example, if we have a set of 100 possible inputs, then there are $2^100$ combinations. Testing for all of these inputs may not be possible. If the input combination are small then we can do all these testing in a very small time. Hence, finding a logical error in an algorithm could be very hard or could be very easy. Second problem may occur when our algorithm is correct but computer is not capable to execute it. It runs in to \textit{numerical problems} e.g. iterrations do not converge or divide by epsilon may happen etc.. Here in these notes, we'll explore some of these algorithms. 
\paragraph*{•}


 