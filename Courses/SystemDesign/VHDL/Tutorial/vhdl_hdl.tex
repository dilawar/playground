\documentclass[a4paper,10pt]{article}
\usepackage[margin=15mm]{geometry}
\usepackage{pgf,tikz}
\usepackage{subfig}
\usepackage{amsmath}
\usepackage{color}
\usepackage{amssymb}
\usepackage{noweb}
\usepackage{listings}
\usetikzlibrary{circuits.logic.US}
\usetikzlibrary{positioning}
\usetikzlibrary{matrix}
 
\definecolor{dkgreen}{rgb}{0,0.6,0}
\definecolor{gray}{rgb}{0.5,0.5,0.5}
\definecolor{mauve}{rgb}{0.58,0,0.82}

\lstset{ %
  language=Verilog,                % the language of the code
  basicstyle=\footnotesize,           % the size of the fonts that are used for the code
  numbers=left,                   % where to put the line-numbers
  numberstyle=\tiny\color{gray},  % the style that is used for the line-numbers
  stepnumber=2,                   % the step between two line-numbers. If it's 1, each line 
                                  % will be numbered
  numbersep=5pt,                  % how far the line-numbers are from the code
  backgroundcolor=\color{white},      % choose the background color. You must add \usepackage{color}
  showspaces=false,               % show spaces adding particular underscores
  showstringspaces=false,         % underline spaces within strings
  showtabs=false,                 % show tabs within strings adding particular underscores
%  frame=single,                   % adds a frame around the code
  rulecolor=\color{black},        % if not set, the frame-color may be changed on line-breaks within not-black text (e.g. commens (green here))
  tabsize=2,                      % sets default tabsize to 2 spaces
  captionpos=b,                   % sets the caption-position to bottom
  breaklines=true,                % sets automatic line breaking
  breakatwhitespace=false,        % sets if automatic breaks should only happen at whitespace
  title=\lstname,                   % show the filename of files included with \lstinputlisting;
                                  % also try caption instead of title
  keywordstyle=\color{blue},          % keyword style
  commentstyle=\color{dkgreen},       % comment style
  stringstyle=\color{mauve},         % string literal style
  escapeinside={\%*}{*)},            % if you want to add LaTeX within your code
  morekeywords={*,...}               % if you want to add more keywords to the set
}
\usepackage{amsthm}
\usepackage{hyperref}
\setlength{\parskip}{3mm}
\newtheorem{axiom}{Axiom}
\newtheorem{definition}{Definition}
\newtheorem{comment}{Comment}
\newtheorem{example}{Example}
\newtheorem{lemma}{Lemma}
\newtheorem{prop}{Property}
\newtheorem{problem}{Problem}
\newtheorem{remark}{Remark}
\newtheorem{theorem}{Theorem}

% Title Page
\title{Introduction to VHDL}
\author{Dilawar Singh}
\date{\today}

\begin{document}
\maketitle

\begin{abstract}
  
  This is not a tutorial of VHDL language. A minimal familiarity with grammar of
  computer languages are assumeed. It has been assumed that you have read the
  document posted on moodle about \textbf{proto-rtl} language.

\end{abstract}

\section{Introduction}
  
 Let's begin with a toy. One excercise is obvious : \emph{Write its equivalent
 proto-RTL description.} Later, we can simply translate this proto-RTL
 description to VHDL (or any other HDL). We can do it by hand or better, we can
 write a computer program to automate the coversion.

 In this tutorial, we'll describe this circuit directly in VHDL.
 
  \begin{figure}[h]
    \input{/home/dilawar/Scripts/tikz/circuit_macros.tex}
    \begin{center}
      \begin{tikzpicture}[circuit logic US]
        %\dflipflop{(c.center)}{2}{2}{``DFF''};
        %% Draw the combinational logic.
        \matrix[column sep=7mm]
        {
          \node (reset) {reset}; & & &  \\
          \node (a) {a}; & \node[and gate] (a1) {}; &  &  \\ 
          \node (b) {b};  & \node[not gate] (n1) {};& \node[or gate] (o1) {}; \\
          \node (clk) {clk};  &  &  \\
        };
        \node (dffNode) at ($(o1.east)+(1,-2)$) {};
        \dflipflop{(dffNode.center)}{1.5}{1.5}{dff};
        \draw (reset) -| (dff_reset);
        \draw (a) -- ++(right:5mm) |- (a1.input 1);
        \draw (b) -- ++(right:5mm) |- (a1.input 2);
        \draw (b) -- ++(right:5mm) |- (n1.input);
        \draw (n1.output) -| (o1.input 2);
        \draw (a1.output) -- ++(right:5mm) |- (o1.input 1);
        \draw (o1.output) -- ++(right:5mm) |- (dff_d);
        \draw (clk) -- ++(right:5mm) |- (dff_clock);
        \draw (dff_q) -- ++(right:10mm) node at ++(0,0) [above] {out};
      \end{tikzpicture}
    \end{center}

    \caption{A small logic circuit. This represents one of the most commonly
    found hardware block : some combinational logic attached to the input of
    flip-flop.  We have D-type flip-flop on the right. Unused pins are
    \texttt{S} (set), $\bar{Q}$, and \texttt{CE} (clock-enable).}
 
    \label{fig:circuit}
 
  \end{figure}

\paragraph{}

  While describing a hardware in VHDL, one needs to know the input and ouput
  (hereafter, \textbf{port}) of the system. We have \texttt{a}, \texttt{b},
  \texttt{clk}, and \texttt{reset} as input to (input port of) this system and
  \texttt{out} as the output (port) of this system shown in figure
  \ref{fig:circuit}. Immediately after this we ask
  the \textbf{type} of ports. We assume that all ports in this system have data-type
  \textbf{bit} \footnote{See the types available in VHDL language. VHDL type
  system is stronger than Verilog. Why should it matter?}.

  Till now, we have the following information about the system :

  \begin{table}[h]
    \centering
  \begin{tabular}{c|c|c}
    Port & Direction & Type \\
    \hline 
    a & in & bit \\
    b & in & bit \\ 
    clk & in & bit \\ 
    reset & in & bit \\ 
    out & out & bit \\
    \hline
  \end{tabular}
\end{table}


\section{Entity-Architecture pair in VHDL}
  
In VHDL, any digital system is described by \textbf{entity-architecture} pair.
\textbf{Entity} describes what a system recieves as input nd what it generates
as output. \textbf{Architecture} describes how the system operates on input to
generate it's output. If an analogy is to be drawn with C/C++ langauges, it can
be said that entity are like function declarations while architecture are like
function definitions. Enough philosophies! Let's see some code.

\lstinputlisting[language=vhdl, firstline=7, lastline=17]{../GHDLDemo/demo.vhd}

And now we have to write the \textbf{architecture} of this entity. There are two
style of writing architecture. One is called \textbf{structural} and other is
\textbf{behavioural}. If we already know the components (as we do for this
system) it is straightforward to write the structural description. When we do
not know the components and their arrangements (in most of the case we do not
know), we write behavioural description. 

\begin{remark}[Structral vs behaviour]
  Structural architecture describes \emph{what is connected with what}. In our
  system we have three gates connected with each other and their output is
  reflected to \texttt{out} at each rising edge of clock (D-flip flop will
  ensure that.). Behavioural architecture describes \emph{what happens when}
  which can be described by the following equation.

  \begin{equation}
  out = 
  \begin{cases}
    (a \land b) \lor \neg b & \quad \text{on rising edge of clock and reset
    is unset}\\
    \text{no change} & \text{otherwise} 
  \end{cases}
  \end{equation}
\end{remark}
\end{document}          
